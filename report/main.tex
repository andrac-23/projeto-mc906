\documentclass[journal]{IEEEtran}


\usepackage[utf8]{inputenc}
\usepackage[portuguese]{babel}
\usepackage{graphicx}
\usepackage{caption}
\usepackage{subcaption}
\usepackage{romannum}

\usepackage[thinc]{esdiff}
\usepackage{float}
\usepackage{graphicx}% Include figure files
\usepackage{dcolumn}% Align table columns on decimal point
\usepackage{hyperref}% add hypertext capabilities
\usepackage{amsmath}
\usepackage[normalem]{ulem}

\usepackage[margin=0.6in]{geometry}
\usepackage{indentfirst}        % indenta primeiro parágrafo

\usepackage{lipsum}

\ifCLASSINFOpdf
  % \usepackage[pdftex]{graphicx}

\else

\fi



\begin{document}

\title{Relatório Técnico de Projeto\\ MC906/MO416 }

\author{
André Rodrigues Alves da Silva (231392) \\
Gustavo Nascimento Soares (217530) \\ 
Marcella de Sant Ana (223588) \\
Vinícius Carvalho Pimpim (194940)
} %AUTORES




% The paper headers
\markboth{MC906/MO416 2024-1, UNICAMP}%
{Shell \MakeLowercase{\textit{et al.}}: Bare Demo of IEEEtran.cls for IEEE Journals}

\maketitle

% As a general rule, do not put math, special symbols or citations
% in the abstract or keywords.
\renewcommand{\abstractname}{Resumo\hspace{0.1cm}}
% \begin{abstract}
% \hspace{0.05cm} \lipsum[1]

% \end{abstract}

\IEEEpeerreviewmaketitle


\section{Introdução}
\label{sec:introducao}

\IEEEPARstart{Q}{uando} \lipsum[1].

\section{Detecção de Objetos}
\label{sec:deteccao_objetos}
\lipsum[1].

\section{YOLO - \textit{You Only Look Once}}
\label{sec:yolo}
\lipsum[1].

\section{Dataset}
\label{sec:dataset}
\lipsum[1].

\section{Testes}
\label{sec:testes}
\lipsum[1].

\section{Treinamento Final}
\label{sec:treinamento_final}
\lipsum[1].

\section{Resultados}
\label{sec:resultados}
\lipsum[1].

\section{Conclusão}
\label{sec:conclusao}
\lipsum[1].

\clearpage

\appendices
\section{Figuras}

\begin{figure}[H]
    \centering
    \includegraphics[width=1\linewidth]{images/ipc.png}
    \caption{Novo tratamento da mensagem \texttt{SCHEDULING\textunderscore INHERIT} em \textit{sched}}
    \label{fig:ipc}
\end{figure}

\begin{figure}[H]
    \centering
    \includegraphics[width=1\linewidth]{images/system-conf.png}
    \caption{Entradas adicionadas em system.conf}
    \label{fig:systemconf}
\end{figure}

\begin{figure}[H]
    \centering
    \includegraphics[width=1\linewidth]{images/track.png}
    \caption{Resultado da execução do programa track\_process\_test}
    \label{fig:track}
\end{figure}

\begin{figure}[H]
    \centering
    \includegraphics[width=1\linewidth]{images/child.png}
    \caption{Resultado da execução do programa child\_test}
    \label{fig:child}
\end{figure}

\begin{figure}[H]
    \centering
    \includegraphics[width=1\linewidth]{images/normal.png}
    \caption{Resultado da execução do programa normal\_test}
    \label{fig:normal}
\end{figure}

% colocar figuras anteriores aqui
\begin{figure}[H]
    \centering
    \includegraphics[width=1\linewidth]{images/panic.png}
    \caption{Mensagem apresentada pelo Minix3 na inicialização do sistema após as alterações}
    \label{fig:panic}
\end{figure}

\begin{thebibliography}{1}

\bibitem{}
Tanenbaum, Andrew S.; Woodhull, Albert S. (1987). Operating Systems: Design and Implementation. ISBN 9780131429383.

\bibitem{}
Minix3 - IPA about the creation of the user mode scheduler, url = https://www.minix3.org/docs/scheduling/report.pdf

\bibitem{}
  Minix3 developers guide - driver programming, url = https://wiki.minix3.org/doku.php?id=developersguide:driverprogramming 

\end{thebibliography}
\end{document}